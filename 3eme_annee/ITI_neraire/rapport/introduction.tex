\chapter{Introduction}

\section{Contexte}
Durant la 3ème année de notre cursus à l'INSA Rouen Normandie, nous devons effectuer un Projet Intégratif (PI). Celui-ci a pour but de mettre en pratique l'ensemble des compétences que nous avons vues en cours. Ici, nous devons mettre en place un système d'acquisition permettant l'estimation de trajectoire d'un parcours. Pour cela, nous pouvons définir un cahier des charges comme suit:
\begin{itemize}
  \item Les capteurs et les données recupérées doivent être en accord avec la finalité du projet (pas de capteurs inutiles et des capteurs biens calibrés);
  \item Le système doit pouvoir récupérer des données exploitables avec une bonne fréquence d'acquisition;
  \item Le système doit effectuer un pré-traitement des données;
  \item Le système doit pouvoir synchroniser les données;
  \item Le système doit fusioner et exploiter les données;
  \item Les résultats doivent pouvoir être enregistrés et affichés sur une carte OpenStreetMap ;
  \item Le système doit proposer des fonctionnalités supplémentaires utiles ou divertissantes;
\end{itemize}

\section{Organisation du rapport}
Afin de vous présenter notre travail, nous découperons ce rapport en plusieurs parties. Dans un premier temps, nous vous présenterons les capteurs que nous avons utilisés ainsi que nos choix pour les paramétrer au mieux. Ensuite, nous vous décrirons le système d’acquisition et vous détaillerons les grandes étapes de la fusion de données. Après cela, nous vous présenterons les analyses des performances que nous avons obtenues ainsi que notre méthode de traitement des données. Pour finir, nous vous présenterons les fonctionnalités supplémentaires que nous avons développées.
