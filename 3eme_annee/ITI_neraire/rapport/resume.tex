\chapter*{Résumé}
\addcontentsline{toc}{chapter}{Résumé}

Ce projet vise à concevoir un système d’acquisition et d’estimation de trajectoire pour activités sportives, dans l’esprit de l’application mobile Strava. L’objectif principal est de permettre à un utilisateur de visualiser son parcours à pied sur une carte, tout en exploitant plusieurs capteurs embarqués pour améliorer la précision de la trajectoire estimée.

Le système repose sur l’utilisation d’une Raspberry Pi couplée à des capteurs tels qu’un GPS, une centrale inertielle (IMU), et un LiDAR. La finalité de ce projet est d'utiliser un ensemble d'outils statistiques ainsi que des connaissances sur les capteurs que nous avons abordés en cours afin de pouvoir calibrer, synchroniser et fusionner au mieux les données issues de ces capteurs.

Par la suite, nous proposerons l'implémentation de fonctionnalités supplémentaires telles que le calcul de distance parcourue, un compteur de pas ou encore des modes de course spécifiques.
