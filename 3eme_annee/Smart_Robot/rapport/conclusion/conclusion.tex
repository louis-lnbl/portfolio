\chapter{Conclusion}

\section{Conclusion globale sur le projet}
De manière plus générale, ce projet nous a permis d'assimiler et de renforcer de nombreuses compétences. En effet, celui-ci utilise un large pannel d'outils et de méthodes de travail indispensable pour un ingénieur. Bien que cette première expérience n'aie pas toujours été parfaitement réalisée, elle aura été très enrichissante. Elle nous aura permis de nous projeter plus concrètement sur le travail que nous aurons à effectuer, et les difficultés que nous rencontrerons, dans les années à venir.

\section{Ressenti personnel de chaque membre \\ du projet}

\begin{itemize}
    \item \textbf{Delaplace Yohann}\\ Le projet m'a permis de lier la théorie et la pratique en la mettant en œuvre sur le robot. Cela m'a aussi permis de m'habituer à utiliser Git afin de réaliser un travail en groupe. J'ai eu l'occasion de programmer en C et en Shell, ce qui m'a permis de comprendre plus en profondeur les notions abordées en cours. Concernant le groupe de travail, chacun avait sa tâche assignée par le chef de projet, avec la date limite à ne pas dépasser pour que le projet puisse bien avancer, et chacun d'entre nous a respecté ce fonctionnement, permettant au projet de se développer plus rapidement. De plus, je tiens à souligner la bonne entente dans notre groupe de travail, ce qui nous a permis de rester motivés du début à la fin et de tout donner pour réussir et avoir un projet qui fonctionne. Je pense que si ce projet a pu être mené à bien, c'est grâce au grand investissement personnel de chacun.
\end{itemize}
\vspace{5mm}

\begin{itemize}
    \item \textbf{Lenoble Louis}\\Bien que le robot ne soit pas encore totalement opérationnel, je suis fier du chemin parcouru par le groupe et de la manière dont chacun a su trouver sa place. Cette complémentarité a créé une dynamique positive et productive, où les idées de chacun enrichissaient le projet. J’ai particulièrement apprécié l’engagement collectif et l’entraide qui se sont développés au fil des séances.\\

Ensuite, ce projet a profondément changé ma manière de percevoir le travail en groupe. J'ai appris à mieux écouter, à anticiper les besoins des autres et à m'adapter aux imprévus. De plus, j'ai pris conscience que l'organisation rigoureuse et la communication ouverte ont été d'une très grande importance pour garantir un avancement régulier et efficace. C’est un apprentissage que je considère précieux, et je me sens désormais bien mieux préparé pour travailler de manière collaborative dans d'autres contextes.\\

Enfin, en tant que chef de projet, cette expérience a été révélatrice. Je me suis rendu compte que ce rôle demande bien plus qu'une simple coordination : il s'agit d'être à l'écoute, de motiver, et de savoir tirer le meilleur de chaque membre du groupe. Organiser les séances, planifier les objectifs et faire des bilans réguliers m'ont demandé beaucoup de temps, mais cela m’a permis de développer une meilleure capacité d’analyse et de décision. Cette immersion dans la gestion de projet m’a permis de grandir, à la fois en tant que leader et en tant qu’individu capable de travailler de façon autonome.
\end{itemize}
\vspace{5mm}

\begin{itemize}
    \item \textbf{Planchot Maël}\\
Ce projet a été une expérience riche et très enrichissante, tant sur le plan technique que personnel. La diversité des tâches à accomplir, ainsi que les imprévus rencontrés, ont mis à l'épreuve notre capacité d'adaptation et notre gestion du stress. Bien que le produit final ait encore des points perfectibles, les résultats obtenus sont encourageants, et je suis satisfait des solutions que nous avons pu apporter. Ce projet m’a permis de me perfectionner en c, de mieux abordés les problèmes de débeugage et d'améliorer ma communication au sein d'une équipe.
\end{itemize}
\vspace{5mm}

\begin{itemize}
    \item \textbf{Sanson Dylan}\\Ce projet a été une expérience à la fois stimulante et formatrice. La complexité des tâches, combinée à la gestion du temps, a été un véritable défi. En tant qu’équipe, nous avons dû nous adapter rapidement, mais la répartition des rôles et l’entraide ont permis de maintenir une bonne dynamique. J’ai particulièrement apprécié de pouvoir mettre en pratique les concepts appris en cours, notamment en programmation C. Bien que le robot ne soit pas totalement fonctionnel, nous avons fait de réels progrès, et je suis fier du travail accompli. Ce projet m’a permis de développer de nouvelles compétences techniques et humaines, en particulier dans le travail en équipe et la gestion de projet.
\end{itemize}
\vspace{5mm}

\begin{itemize}
    \item \textbf{Sourdrille Nathan}\\
    Ce projet était une première grande expérience pour moi de part la quantité de travail à fournir, l'étendu de la durée du projet, l'utilisation de nouveaux outils et le travail en groupe de cinq. D'abord, la quantité de travail à fournir était très importante, et cela bien que nous ayons réparti, il me semble, plutôt bien le travail. Ce projet s'est étendu sur deux mois mais quelques jours de plus aurait été grandement apprécié, afin de résoudre les problèmes rencontrés lors de la phase de test, que nous n'avons pu effectuer que durant la dernière semaine. D'autre part, je n'avais jamais utilisé, hormis leur introduction durant les séances de td, les outils gitlab, latex ... C'était aussi une première expérience de vrai programmation en C. Cependant c'était très intéressant d'exploiter ce que nous avons pu voir dans les différentes matières, et de pousser plus loin la SE par exemple, en naviguant dans le terminal, créant des scripts bash ... Globalement, le projet est très intéressant et très satisfaisant quand on arrive finalement à faire sortir le robot du labyrinthe, cependant il demande énormément de travail personnel, de l'organisation et une entente harmonieuse entre les membres du groupe est indispensable pour mener le projet à son terme.
\end{itemize}





