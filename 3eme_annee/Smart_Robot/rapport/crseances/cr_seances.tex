\documentclass[a4paper,12pt]{article}
\usepackage[utf8]{inputenc} % Encodage UTF-8
\usepackage[T1]{fontenc}    % Gestion des accents
\usepackage{geometry}       % Gestion des marges
\usepackage{setspace}       % Espacement des lignes
\usepackage{hyperref}       % Liens hypertextes

% Configuration de la mise en page
\geometry{margin=1in}
\setstretch{1.15}

\begin{document}

\title{Compte-rendu des séances}
\author{}
\date{}
\maketitle

\section*{COMPTE-RENDU DE CHAQUE SÉANCE}

\textbf{Note} : J'ai forcément oublié des éléments que vous avez réalisés durant les séances, car je faisais autre chose. N'hésitez pas à me le dire.

\section{TP ELEC PROJET}

\subsection{Séance 1 (16/10)}
\begin{itemize}
    \item Réflexion sur la construction du robot :
    \begin{itemize}
        \item Placement des moteurs, des capteurs.
        \item Distances entre les éléments.
        \item Nombre de capteurs nécessaires.
    \end{itemize}
    \item Réflexion sur les stratégies de déplacement :
    \begin{itemize}
        \item Méthodes pour prendre des virages et avancer efficacement.
        \item Solutions retenues :
        \begin{itemize}
            \item Utilisation de 4 capteurs : 3 à l'avant (1 centré, 1 à gauche, 1 à droite) et 1 à l'arrière (centré avec celui à l'avant).
            \item Objectif : sortir du labyrinthe le plus vite possible.
        \end{itemize}
    \end{itemize}
    \item Méthode de virage :
    \begin{itemize}
        \item Le robot continue d'avancer tout en tournant en courbe.
        \item Le capteur arrière détecte la ligne sur laquelle il était avant le virage.
        \item Une fois la ligne détectée, le robot se recentre pour finir le virage.
        \item En cas de problème, une rotation classique de 90° pourra être utilisée.
    \end{itemize}
\end{itemize}

\subsection{Séance 2 (06/11)}
\begin{itemize}
    \item \textbf{Louis} : Organisation et branchement sur la carte Raspberry (à l'exception des moteurs). \texttt{push} du schéma Fritzing.
    \item \textbf{Yohann/Maël} : Découpage d'un modèle de flamme imprimé, collage sur le robot, perçage et peinture des murs en bleu et rouge.
    \item \textbf{Nathan/Dylan} : 
    \begin{itemize}
        \item Clonage du dépôt Git sur la carte.
        \item Séparation des fichiers et dossiers.
        \item Écriture d'un algorithme pour les déplacements du robot (\texttt{parcours\_laby.c}).
        \item Simplification de l'analyse descendante (fusion de fonctions redondantes comme \texttt{RotationGauche} et \texttt{RotationDroite}).
        \item Implémentation de l'arrêt d'urgence : vitesse des moteurs mise à 0 et extinction.
    \end{itemize}
    \item Détails du fonctionnement du robot :
    \begin{itemize}
        \item \textbf{Se remettre droit} : Si un capteur latéral détecte une ligne mais pas les autres.
        \item \textbf{Virages à gauche/droite} : Détection par le capteur central et un capteur latéral.
        \item \textbf{Avancer} : Avancer à vitesse modérée pour limiter les arrêts.
        \item \textbf{Sortie du labyrinthe} : Si aucun capteur ne détecte de ligne, le robot coupe les moteurs.
    \end{itemize}
    \item Tests des moteurs et des roues, impression des motifs du robot.
\end{itemize}

\subsection{Séance 3 (20/11)}
\begin{itemize}
    \item \textbf{Louis/Yohann} : Branchement des moteurs, montage du robot (fixation des éléments, soudures, etc.).
    \item \textbf{Dylan/Nathan/Maël} : Écriture du code en C pour les moteurs, l'écran LCD, le signal sonore, et le buzzer.
    \item Discussion sur l'arrêt d'urgence en virage.
    \item Préparation pour la séance suivante :
    \begin{itemize}
        \item Finaliser le code pour que tout fonctionne.
        \item Compiler et tester le parcours du labyrinthe avec les threads.
    \end{itemize}
\end{itemize}

\subsection{Séance 4 (04/12)}
\begin{itemize}
    \item Objectifs : Vérification des fonctions, tests sur le robot.
\end{itemize}

\section{TD ALGO PROJET}

\subsection{Séance 1 (22/10)}
\begin{itemize}
    \item \textbf{Objectif} : Mise en place du projet GitLab et spécification du TAD \texttt{Labyrinthe}.
    \item \textbf{Travail réalisé} : FAIT avant la séance 2.
\end{itemize}

\subsection{Séance 2 (05/11)}
\begin{itemize}
    \item \textbf{Objectif} : Analyse descendante des opérations suivantes :
    \begin{itemize}
        \item \texttt{Labyrinthe → Liste<NaturelNonNul>}
        \item \texttt{Liste<NaturelNonNul> → Liste<Ordre>}
    \end{itemize}
    \item \textbf{Travail réalisé} : FAIT avant la séance 3.
\end{itemize}

\subsection{Séance 3 (13/11)}
\begin{itemize}
    \item \textbf{Objectif} : Conception préliminaire et détaillée.
    \item \textbf{Travail réalisé} :
    \begin{itemize}
        \item \textbf{Louis} : CP + CD Trouver le plus court chemin, écriture du TAD \texttt{Noeud}.
        \item \textbf{Dylan/Nathan} : CP + CD Conversion des noeuds en ordres.
        \item \textbf{Maël} : CP + CD Analyse de fichier et macros.
        \item \textbf{Yohann} : CP + CD Création de passage.
    \end{itemize}
\end{itemize}

\subsection{Séance 4 (19/11)}
\begin{itemize}
    \item \textbf{Objectif} : Développement des fichiers \texttt{.h}, des fichiers \texttt{.c} vides, et des tests unitaires.
    \item \textbf{Travail réalisé} :
    \begin{itemize}
        \item Clarification de l'analyse descendante et des principes des algorithmes.
        \item Mise en ordre des conventions de codage (nommage des fonctions, accolades, etc.).
        \item Discussions sur le TAD \texttt{Noeud} : suppression et modifications nécessaires.
        \item Organisation des bibliothèques et fichiers.
    \end{itemize}
\end{itemize}

\subsection{Séance 5 (26/11)}
\begin{itemize}
    \item \textbf{Objectif} : Développement des fichiers \texttt{.c}.
\end{itemize}

\subsection{Séance 6 (03/12)}
\begin{itemize}
    \item Objectifs et travail à définir.
\end{itemize}

\end{document}
